\chapter{The Belvedere Glacier}

\newthought{This chapter is based on:} 

or having a good opening line. Morbi commodo, ipsum sed pharetra gravida, orci  $x = 1/\alpha$ magna rhoncus neque, id pulvinar odio lorem non turpis \cite{Eigen1971, Knuth1968}. Nullam sit amet enim. Suspendisse id velit vitae ligula volutpat condimentum. Aliquam erat volutpat. Sed quis velit. Nulla facilisi. Nulla libero. Vivamus pharetra posuere sapien.

\newpage

Donec porttitor ligula eu dolor. Maecenas vitae nulla consequat libero cursus venenatis. Nam magna enim, accumsan eu, blandit sed, blandit a, eros.
$$\zeta = \frac{1039}{\pi}$$


% For an example of a full page figure, see Fig.~\ref{fig:myFullPageFigure}.


Proin at eros non eros adipiscing mollis. Donec semper turpis sed diam. Sed consequat ligula nec tortor. Integer eget sem. Ut vitae enim eu est vehicula gravida. Morbi ipsum ipsum, porta nec, tempor id, auctor vitae, purus. Pellentesque neque. Nulla luctus erat vitae libero. Integer nec enim. Phasellus aliquam enim et tortor. Quisque aliquet, quam elementum condimentum feugiat, tellus odio consectetuer wisi, vel nonummy sem neque in elit. Curabitur eleifend wisi iaculis ipsum. Pellentesque habitant morbi tristique senectus et netus et malesuada fames ac turpis egestas. In non velit non ligula laoreet ultrices. Praesent ultricies facilisis nisl. Vivamus luctus elit sit amet mi. Phasellus pellentesque, erat eget elementum volutpat, dolor nisl porta neque, vitae sodales ipsum nibh in ligula. Maecenas mattis pulvinar diam. Curabitur sed leo.

\texttt{This is a line of code.}


\begin{figure}
\includegraphics[width=\textwidth]{figures/fig1}
\caption[Short figure name.]{This is a figure that floats inline and here is its caption.
\label{fig:myInlineFigure}}
\end{figure}


%% Requires fltpage2 package
%%
% \begin{FPfigure}
% \includegraphics[width=\textwidth]{figures/fullpage}
% \caption[Short figure name.]{This is a full page figure using the FPfigure command. It takes up the whole page and the caption appears on the preceding page. Its useful for large figures. Harvard's rules about full page figures are tricky, but you don't have to worry about it because we took care of it for you. For example, the full figure is supposed to have a title in the same style as the caption but without the actual caption. The caption is supposed to appear alone on the preceding page with no other text. You do't have to worry about any of that. We have modified the fltpage package to make it work. This is a lengthy caption and it clearly would not fit on the same page as the figure. Note that you should only use the FPfigure command in instances where the figure really is too large. If the figure is small enough to fit by the caption than it does not produce the desired effect. Good luck with your thesis. I have to keep writing this to make the caption really long. LaTex is a lot of fun. You will enjoy working with it. Good luck on your post doctoral life! I am looking forward to mine. \label{fig:myFullPageFigure}}
% \end{FPfigure}
% \afterpage{\clearpage}
