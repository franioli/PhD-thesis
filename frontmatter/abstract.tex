Alpine glaciers are key indicators of global climate change, experiencing rapid transformations due to the climate crisis. 
Understanding these changes is essential for assessing climate impacts, predicting future trends, and mitigating related hazards.
This thesis comprehensively investigates multi-scale and multi-temporal photogrammetry for monitoring alpine glaciers, focusing on the debris-covered Belvedere Glacier in the Italian Alps.

This research employs various photogrammetric techniques, from historical aerial image analysis to high-resolution UAV surveys and a novel low-cost stereo camera system for daily 3D motion analysis.
Archival aerial images have been used to reconstruct the evolution of Belvedere Glacier over the last 50 years. 
This analysis revealed a period of expansion until a surge event in the early 21st century, followed by a significant retreat in recent decades.
Recent annual UAV surveys have provided 3D glacier reconstructions with sub-decimetre accuracy, allowing quantification of ice volume loss and determination of glacier flow kinematics through digital image correlation.

Recognizing the inherently non-linear nature of glacier dynamics, this thesis presents a low-cost stereoscopic system built with homemade time-lapse cameras controlled by microcontrollers for deriving daily 3D models of the glacier terminal lobe by photogrammetry.
An automated workflow, powered by state-of-the-art deep learning feature matching algorithms, overcomes the challenges posed by the wide baselines of the stereo cameras and has resulted in the derivation of daily 3D displacements and glacier melt.
This has allowed for correlating the glacier kinematics and ice melting with external factors such as air temperature, highlighting a strong short-term correlation between the three variables. 

A commitment to open science underpins this research. 
Datasets, including point clouds, orthophotos, and DSMs, are publicly available to encourage collaboration and future research. 
The software tools developed, ICEpy4D and Deep-Image-Matching, are released as open source software for wider use and are easily adaptable to other photogrammetry-based monitoring applications.

This thesis demonstrates that photogrammetry is a versatile and effective tool for monitoring alpine glaciers and deriving valuable 3D information across different spatial and temporal scales. 
The high-resolution data acquired and the methods developed enhance our understanding of the effects of climate change in alpine environments and can be used to monitor their evolution and refine glaciological models.

\vspace{5pt}
\newthought{\textbf{Keywords:}} Belvedere Glacier, 3D reconstruction, UAV, historical images, Deep-Learning photogrammetry, low-cost cameras, image correlation, feature matching, climate change, open data.