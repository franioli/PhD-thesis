Alpine glaciers are sensitive indicators of global climate change and are undergoing rapid changes due to the ongoing climate crisis. 
Understanding these changes is crucial for assessing climate impacts, predicting future trends, and mitigating related hazards.  
This thesis presents a comprehensive investigation into the application of multi-scale and multi-temporal photogrammetry for in-depth monitoring of alpine glaciers, with a focus on the debris-covered Belvedere Glacier in the Italian Alps.

The research uses various photogrammetric techniques, ranging from historical aerial image analysis to reveal long-term evolution to high-resolution UAV surveys for precise annual monitoring and a novel low-cost stereo camera system for daily 3D motion analysis.
These approaches provide insights into glacier dynamics across multiple spatial and temporal scales.

Analysis of archival aerial imagery revealed a period of glacier expansion up to a surge event that occurred in the early 21st century, followed by a significant retreat in recent decades. 
Annual UAV surveys with sub-decimeter accuracy provide detailed quantification of ice volume loss and dynamic flow patterns, establishing a valuable benchmark for validating surface displacement measurements.

Recognizing the inherently non-linear nature of glacier dynamics, this thesis pioneered a daily 3D displacement analysis system powered by deep learning.  
This represents a significant advance in glacier monitoring at close range, overcoming the challenges of wide baselines in stereo camera image pairs.

A commitment to open science underpins this research.  
Datasets, including point clouds, orthophotos, and DSMs, are publicly available to foster collaboration and future research.  
In addition, the software tools developed (ICEpy4D, Deep-Image-Matching) will be published online for wider use. 
These resources have the potential to transform glacier monitoring and can be adapted to other geophysical mass movements.

This thesis contributes to our understanding of alpine glacier behavior under climate change pressures and provides transformative tools for advancing glacier monitoring. 
The high-resolution data and methods developed have the potential to inform both localized glacier modeling and contribute to our broader understanding of climate change impacts in alpine environments.