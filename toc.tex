%!TEX TS-program = xelatex 
%!TEX encoding = UTF-8 Unicode

% Modify the following line to match your school
% Available options include `Harvard`, `Princeton`, and `NYU`.
\documentclass[12pt]{article}
\usepackage{outlines}
\usepackage[left=1cm,right=1cm,vmargin=2cm]{geometry}
\usepackage{enumitem}
\usepackage{xcolor}

\setlist[1]{label=\arabic*}
\setlist{label*=.\arabic*}

\begin{document}

Introduction on importance of glacier monitoring, strenght and limits of different
techniques used (platforms, scales ecc..),
Aim of the thesis:

Story:

- start with UAV photogrammetry: traditional techniques with well-established processing
procedure to reconstruct glacier geometry.
Outputs: annual pcd, dem, orthotophotos which allows for estimating annual average
glacier flow, volume variations, glacier retreat (compute glacier outline).
Temporal frequency: annual. Spatial scale: high (cm/dm), all glacier

- How to go back in time and reconstruct the history of a glacier: digitalization of
historical aerial images acquired from mapping purposed. Still established approaches.
Only way to build accurate DSM for past years.
Outputs: same as UAV-photogrammetry. Temporal frequency: 10-years. Spatial scale:
sub-meter, all glacier (and more)

- Glaciers are rapidily moving and there is the need for high-frequency monitoring: fixed
time lapse cameras. Challange of wide baseline hindered reconstruction with traditional
photogrammetry. Developing ad hoc workflow based on deep-learning photogrammetry.
Outputs: daily velocity, volume variation and glacier retreat. Temporal frequency: daily.
Spatial scale high:

- Is it possible to use HR stereo satellite to build coarse but accurate DSM, without
in-situ measurements?
Case of debris-flow of august 2023.

- Discussion:
Combining different techniques gives a comprehensive overview of the phenomenon.
Correlation with environmental data at different temporal scale.
Powerful of new DL techniques for solving challanging problems.
....

\begin{outline}[enumerate]
    \1 Introduction: photogrammetry for multi-temporal monitoring of mountain env.
    \2 Motivation and relevance
    \2 The Belvedere Glacier
    \2 Low-cost photogrammery
    \2 Deep learning photogrammetry for solving challenging situations
    \2 Aims of the thesis
    \2 Thesis outline
    \1 Back to the past: reconstructing glacier geometry with historical images
    (1977-2009)
    \2 Introduction
    \2 Photogrammetric Dataset
    \3 Historical Aerial Datasets of 1977, 1991 and 2001
    \3 Digital Aerial and UAV Datasets of 2009 and 2019
    \3 The GNSS Survey for Block Georeferencing
    \2 Methods
    \3 Image Orientation and Point Cloud Generation
    \3 Digital Surface Model Preliminary Processing
    \3 Glacier outline
    \2 Discussion
    \2 References
    \1 UAV photogrammetry (2015-2023) {\color{red} (note: need to add 2021-2022-2023 data
            and PIV on orthophotos)}
    \2 Introduction {\color{red}(tenere intro per ogni capitolo? Magari senza sezione
            separata)}
    \2 Instruments and datasets
    \3 UAV flights
    \3 GNSS measurements
    \2 Methodology
    \3 SfM workflow
    \3 Glacier flow velocity
    \4 GNSS velcoities
    \4 DIC on orthotphotos \textcolor{red}{(TO DO)}
    \4 MAN points (for first years and validation of DIC)
    \3 Volume variations
    \4 Cross sections
    \4 Global ice volume loss
    \3 Glacier outline \textcolor{red}{(TO DO)}
    \2 Results
    \3 SfM
    \3 Glacier velocity
    \3 Volume variation
    \3 Glacier outline \textcolor{red}{(TO DO)}
    \2 Discussion (Comparison with previous studies)
    \2 References
    \1 Short-term monitoring with low-cost fixed time-lapse cameras and deep-learning
    stereo photogrammetry
    \2 Introduction
    \2 The low-cost stereoscopic system {\color{red} (paper antalya + last paper)}
    \3 System description
    \4 The circuit
    \4 Power supply
    \4 System control and scheduling
    \4 Connectivity
    \4 Case and protection
    \4 General performances
    \3 Camera and lenses
    \3 Monumentation
    \2 Datasets and methods
    \3 Datasets
    \4 Image sequences
    \4 GCPs
    \4 UAV surveys	{\color{red} (Add 2023 UAV blocks)}
    \4 Meteorological monitoring station
    \3 Image selection
    \3 Camera calibration
    \3 Camera stability and GCPs
    \3 Stereoscopic image processing workflow
    \4 Wide-baseline image matching
    \4 Tracking points over epochs
    \4 3D scene reconstruction
    \3 Volume variation estimation
    \3 Automatic extraction of ice cliff top edge
    \3 Digital Image Correlation from single cameras
    \4 Orthorectification uncertainty
    \3 Correlation between glacier dynamics and meteorological variables
    \2 Results {\color{red} (Add 2023 sequence!)}
    \3 Image acquisition
    \3 Automatic detection of GCPs
    \3 Wide-baseline feature matching and tracking
    \3 3D scene reconstruction
    \3 Volume variations and glacier retreat
    \3 Validation of the stereo models with UAV data
    \3 Glacier surface velocity and morphology
    \3 Velocity orthorectification uncertainty
    \3 Comparison between surface velocity, frontal ice loss and meteorological
    variables
    \2 Discussion
    \3 Hand-crafted vs deep learning matching
    \3 Merging stereoscopic and monoscopic processings to study the glacier
    dynamics
    \3 Glacier velocity, frontal ablation and temperature
    \3 Transferability of the system
    \2 References
    \1 High-res satellite stereo photogrammetry  \textcolor{red}{(TO DO, if there is
        time)}
    \1 Discussion
    \2
    \1 Conclusions
    \1 Appendixes:
    \2 Published papers
    \3 RS historical
    \3 RS historical
    \3 PFG
    \2 Open software
    \3 ICEpy4D (paper ICEpy4D)
    \3 Deep-image-matching (paper 3Darch)

\end{outline}

\end{document}